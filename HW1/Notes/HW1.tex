\documentclass{homework}

\linespread{1.2}

\usepackage[T1]{fontenc}\usepackage{palatino}
\usepackage{amssymb,amsmath,amsthm,mathtools}
\usepackage[english]{babel}
\usepackage[utf8]{inputenc}
\usepackage[babel]{csquotes}
\usepackage[pdftex]{hyperref}
\usepackage{enumitem}
\usepackage[usenames,dvipsnames,table,xcdraw]{xcolor}

\usepackage{graphicx} 
\usepackage{verbatim} % Commenti in blocco con \begin{comment}
\usepackage{bm}
\usepackage[font={small,it}]{caption}
\usepackage{subcaption}
\usepackage{geometry}
\usepackage{array}
\usepackage{enumitem}
\setlist[enumerate]{label*=\arabic*.}
	
\usepackage{algorithm}
\usepackage[noend]{algpseudocode}
\usepackage{listings}

\definecolor{mygraybackground}{gray}{0.95}
\definecolor{listinggray}{gray}{0.9}
\definecolor{lbcolor}{rgb}{0.9,0.9,0.9}

\lstset{language=C++,
backgroundcolor=\color{lbcolor},
    tabsize=3,    
        basicstyle=\scriptsize\ttfamily,
        showstringspaces=false,
        breaklines=true,
        frame=single,
        numbers=left,
        showtabs=false,
        showspaces=false,
        showstringspaces=false,
        identifierstyle=\ttfamily,
        keywordstyle=\color{blue},
        captionpos=b,   
        commentstyle=\color{ForestGreen},
        stringstyle=\color[rgb]{0.627,0.126,0.941},
        numberstyle=\small\ttfamily\color{Gray}
}


\makeatletter
\def\BState{\State\hskip-\ALG@thistlm}
\makeatother

% Simbolo iid
\newcommand\iid{\stackrel{\mathclap{\normalfont\tiny\mbox{iid}}}{\sim}}
% Simbolo ind
\newcommand\ind{\stackrel{\mathclap{\normalfont\tiny\mbox{ind}}}{\sim}}
\DeclareMathOperator*{\argmin}{argmin}
\DeclareMathOperator*{\sign}{sign}


\algdef{SE}[DOWHILE]{Do}{doWhile}{\algorithmicdo}[1]{\algorithmicwhile\ #1}%

\title{SDS 383D: Homework 1}
\author{Giorgio Paulon}

\begin{document}

\makeatletter
\begin{titlepage}
	\vspace*{\fill}
	\centering
	{\huge \@title \par}
	\vskip0.5cm
	{\large \@author \par}
	\vskip0.5cm
	{\large \today \par}
	\vspace*{\fill}
\end{titlepage}
\makeatother

\newpage 
\mbox{}
\thispagestyle{empty}
\newpage

\setcounter{page}{1}

\problem{Bayesian inference in simple conjugate families}

We start with a few of the simplest building blocks for complex mul- tivariate statistical models: the beta/binomial, normal, and inverse- gamma conjugate families.

\begin{enumerate}[label=(\Alph*)]
\item Suppose that we take independent observations $X_1, \dots , X_N$ from a Bernoulli sampling model with unknown probability $w$. That is, the $X_i$ are the results of flipping a coin with unknown bias. Suppose that $w$ is given a $\text{Beta}(a,b)$ prior distribution:
$$p(w) = \Gamma(a + b) w^{a-1} (1 - w)^{b-1},$$
where $\Gamma(\cdot)$ denotes the Gamma function. Derive the posterior distribution $p(w | x_1,\dots,x_N)$.



\item The probability density function (PDF) of a gamma random variable, $X \sim Ga(a, b)$, is


Suppose that x
new random variables y1 = x1/(x1 + x2) and y2 = x1 + x2. Find the joint density for (y1, y2) using a direct PDF transformation (and its Jacobian).2 Use this to characterize the marginals p(y1 ) and p(y2 ), and propose a method that exploits this result to simulate beta random variables, assuming you have a source of gamma random variables.



\item Suppose that we take independent observations $X_1, \dots , X_N$ from a normal sampling model with unknown mean $\theta$ and known variance $\sigma^2$: xi ∼ N(θ, σ2). Suppose that θ is given a normal prior distribu- tion with mean m and variance v. Derive the posterior distribution p(θ | x1,...,xN).

\end{enumerate}

\clearpage

\appendix
\chapter{R code}
\label{chap:code}




\end{document}
